\documentclass[journal]{IEEEtran}
\usepackage[T1]{fontenc}
\usepackage[utf8]{inputenc}
\usepackage[polish]{babel}
\usepackage{graphicx}
\pagestyle{plain}
\setlength{\parskip}{0.7em}  
\setlength{\parindent}{15pt}


\ifCLASSINFOpdf
  
\else
 
\fi





% correct bad hyphenation here
\hyphenation{op-tical net-works semi-conduc-tor}


\begin{document}

\title{Analiza zmienności rytmu serca z wykorzystaniem urządzeń mobilnych}
\author{
    J. Bancerewicz, J. Kotłowski, O. Lozovyy, J. Morawska, M. Rzęsa\\
    \textit{Politechnika Gdańska}\\
    \textit{Wydział Elektroniki, Telekomunikacji i Informatyki}
}


% The paper headers
\markboth{}%
{Shell \MakeLowercase{\textit{et al.}}: Bare Demo of IEEEtran.cls for IEEE Journals}
\maketitle




\section{Wstęp}

%\IEEEPARstart{T}{his} demo file is intended to serve as a ``starter file''
%for IEEE journal papers produced under \LaTeX\ using
%IEEEtran.cls version 1.8b and later.

\newpage
\section{Techniki filtracji i modelowania sygnałów biomedycznych}

Metody uczenia maszynowego znajdują szerokie zastosowanie w przetwarzaniu sygnałów biomedycznych, w tym w elektrokardiografii oraz fotopletyzmografii. Techniki te umożliwiają automatyczną detekcję istotnych cech, ocenę długotrwałych zależności czasowych oraz szacowanie parametrów fizjologicznych. W literaturze wyróżnia się dwa główne podejścia: metody oparte na filtracji sygnałów i analizie w dziedzinie czasu oraz modele uczące się, zdolne do samodzielnego wyodrębniania wzorców z danych pomiarowych.

Analiza sygnałów elektrokardiograficznych coraz częściej wykorzystuje metody uczenia maszynowego, jak sieci splotowe czy rekurencyjne, ze względu na możliwość detekcji złożonych zależności czasowych. W związku z wysoką odpornością na szumy i zakłócenia modele ML implementowane są do automatycznej detekcji załamków R. W opracowaniach opisywane są również klasyczne metody, takie jak filtry progowe czy transformacje sygnału. Ich zastosowanie w warunkach rzeczywistych jest utrudnione spadkiem efektywności w obecności zniekształceń ruchowych. W odpowiedzi na ograniczenia, modele głębokiego uczenia stanowią alternatywę dla standardowych metod analizy danych \cite{1}.

Algorytm Pan-Tompkins, szeroko stosowany w przetwarzaniu EKG, wykazuje ograniczoną skuteczność przy niskiej jakości danych. Zaawansowane metody wykorzystują architektury splotowe do ekstrakcji lokalnych cech sygnału oraz sieci rekurencyjne do detekcji zależności czasowych, opisujących relacje między kolejnymi interwałami RR, istotnych dla analizy zmienności rytmu serca \cite{2}. Natomiast modele Transformer umożliwiają jednoczesną identyfikację dynamiki zapisu, zwiększając jego odporność na zakłócenia.  Architektury hybrydowe łączące sieci CNN i RNN osiągają wyższą skuteczność w klasyfikacji nieregularnych przebiegów EKG oraz poprawiają dokładność analizy HRV, jednocześnie minimalizując wady pojedynczych rozwiązań \cite{3}.

\newpage
Analogicznie do sygnałów elektrokardiograficznych, w fotopletyzmografii obserwuje się wykorzystanie algorytmów uczenia maszynowego w monitorowaniu i analizie częstości pracy serca. Wysoka podatność zapisu PPG na zakłócenia ruchowe oraz zmiany oświetlenia ogranicza skuteczność metod filtracji i detekcji. Zastosowanie uczenia głębokiego umożliwia precyzyjne wykrywanie szczytów fali, poprawiając wiarygodność klasyfikacji \cite{4}.

W przetwarzaniu sygnałów PPG wykorzystuje się filtrację pasmową oraz algorytmy detekcji lokalnych maksimów i progów, uwzględniające adaptację do zmian amplitudy. Pomimo wysokiej skuteczności w warunkach kontrolowanych, metody te charakteryzują się obniżoną efektywnością w obecności nasilonych zakłóceń. W nowoczesnej analizie stosuje się architektury głębokiego uczenia, w których warstwy konwolucyjne łączone są z rekurencyjnymi LSTM umożliwiając modelowanie zarówno krótkookresowych, jak i długoterminowych zmian przebiegu fali \cite{5}. W procesie klasyfikacji wyodrębnionych cech implementuje się również wielowarstwowe perceptrony MLP oraz sieci GRU, które przy mniejszej liczbie parametrów wykrywają zależności czasowe z dokładnością zbliżoną do architektury LSTM.
Z sygnału PPG wyznacza się interwały międzyuderzeniowe IBI, odpowiadające interwałom RR w EKG, pozwalając na analizę zmienności rytmu serca niezależnie od zapisów elektrokardiograficznych. W wybranych rozwiązaniach wykorzystuje się uczenie transferowe, umożliwiające dostosowanie modeli wytrenowanych na odmiennych zbiorach danych, zapewniając wyższą precyzję detekcji oraz odporność na zakłócenia \cite{6}.


\subsection{Elektrokardiografia}
Jednym z elementów zaprojektowanych w ramach niniejszej pracy jest sieć łącząca warstwy splotowe oraz rekurencyjne, przeznaczoną do rozpoznawania szczytów R w sygnale EKG. Stanowi ona pierwszy etap w procesie wyznaczania interwałów RR oraz parametrów zmienności rytmu serca.

Model przetwarza jednowymiarowe sygnały napięcia elektrycznego, podzielone na fragmenty o długości 256 próbek. Moduły splotowe odpowiadają za ekstrakcję cech z danych wejściowych przez zastosowanie kolejnych warstw konwolucyjnych wraz z nieliniowymi funkcjami aktywacji. Realizacja procesu redukcji wymiarowości przez wybór największych wartości pozwala na zwiększenie odporności na szum oraz zakłócenia. Po przekształceniu sygnału przez część konwolucyjną, dane przekazywane są do jednokierunkowej warstwy LSTM, umożliwiającej analizę zależności czasowych między kolejnymi próbkami sygnału. W końcowej części modelu wykorzystano połączone moduły liniowe, których celem jest konwersja wewnętrznych cech w przestrzeń logitów. Każdy element wektora reprezentuje prawdopodobieństwo wystąpienia załamka R w odpowiadającej mu próbce sygnału wejściowego. Zaprojektowane rozwiązanie umożliwia binarną klasyfikację dla każdego punktu czasowego.

\newpage
Model został wytrenowany w trybie nadzorowanym na podstawie sygnałów pochodzących z czujnika tętna Polar H10. Dla danych treningowych wykorzystano funkcję ecg\_peaks z biblioteki NeuroKit2, implementującą algorytm Pan–Tompkins, umożliwiający detekcję załamków R w przefiltrowanym zapisie EKG. Dla każdej próbki przygotowano etykiety binarne wskazujące obecność lub brak lokalnego maksimum. W konsekwencji sieć neuronowa uczyła się klasyfikacji poprzez identyfikację wzorców odpowiadającym rzeczywistym szczytom R.


\subsection{Fotopletyzmografia}
Drugim opracowanym rozwiązaniem jest sieć wykorzystująca warstwy splotowe, przeznaczona do analizy sygnału fotopletyzmograficznego. Model ten odpowiada za detekcję lokalnych szczytów, na podstawie których wyznaczane są interwały międzyuderzeniowe, służące do szacowania wskaźników rytmu serca.

System przetwarza jednowymiarowe sygnały reprezentujące zmiany objętości krwi w naczyniach obwodowych, podzielone na segmenty o długości 50 próbek. Kolejne moduły splotowe, wykorzystujące nieliniowe funkcje aktywacji, odpowiadają za identyfikację lokalnych wzorców w danych wejściowych. Selekcja wartości o największej amplitudzie zwiększa odporność na niepożądane zakłócenia. Na wyjściu modelu generowany jest wektor prawdopodobieństw, otrzymany przez zastosowanie funkcji sigmoidalnej do surowych sygnałów warstwy konwolucyjnej. Uzyskane wartości odzwierciedlają lokalizacje szczytów fali, umożliwiając klasyfikację binarną próbek w reprezentacji czasowej.


Uczenie nadzorowane, zastosowane w analizie sygnału EKG pozyskanego z pulsometru, zostało analogicznie zaimplementowane dla danych PPG rejestrowanych za pomocą urządzenia mobilnego. Proces etykietowania oparto na identyfikacji szczytów lokalnych, wykorzystując funkcję find\_peaks z modułu scipy.signal, w przefiltrowanym przebiegu. Każdej próbce przypisano wartość binarną, odzwierciedlającą obecność bądź brak piku fali. Na podstawie zdefiniowanych etykiet sieć neuronowa została wytrenowana do rozpoznawania wzorców odpowiadających detekcji szczytów.

\newpage
\section{Opis systemu i danych}
\subsection{Akwizycja danych}
\subsubsection{Akwizycja danych EKG}
Dane wykorzystane do trenowania i walidacji modelu detekcji załamków R zostały pozyskane za pomocą pulsometru Polar H10, zdolnego do rejestrowania sygnału elektrokardiograficznego z częstotliwością próbkowania wynoszącą 130 Hz. Przyjęta wartość umożliwia charakterystykę przebiegu istotną dla analizy HRV.

Sensor Polar H10 jest czujnikiem tętna stosowanym w sporcie oraz diagnostyce. Działanie urządzenia opiera się na elektrodach piersiowych rejestrujących potencjały elektryczne związane z aktywnością serca, zapewniając wyższą precyzję pomiaru w porównaniu z metodami optycznymi. Transmisja przebiegu realizowana jest poprzez standard Bluetooth Low Energy, a pamięć wewnętrzna umożliwia zapis danych w trybie offline. Dokładność detekcji Polar H10 jest zbliżona do systemów EKG jednokanałowych stosowanych w diagnostyce klinicznej, umożliwiając zastosowanie w mobilnym rejestrowaniu sygnału \cite{7}.

W celu zgromadzenia odpowiedniego zbioru danych przeprowadzono dwugodzinne eksperymenty pomiarowe z udziałem pięciu osób. Rejestrowane zapisy były zróżnicowane pod względem poziomu aktywności fizycznej, zmienności rytmu serca, a także obecnością nagłych ruchów ciała, prowadzących do zakłóceń przebiegu. Sygnał przesyłano w czasie rzeczywistym w pakietach po 13 punktów zgodnych z przyjętą częstotliwością próbkowania i zapisywano w formacie CSV. 


\subsubsection{Akwizycja danych PPG}
\paragraph{Aplikacja mobilna}
Zaprojektowano oprogramowanie w systemie Android realizujące pomiar tętna metodą fotopletyzmografii. Rejestrację sygnału przeprowadzono poprzez umieszczenie opuszka palca bezpośrednio na obiektywie kamery oraz zintegrowanym źródle LED, umożliwiając detekcję zmian optycznych wywołanych cyklicznymi wahaniami objętości krwi w tkankach skórnych.

Aplikacja prezentuje przebieg PPG w czasie rzeczywistym w postaci dynamicznego wykresu zmian rytmu serca. Zarejestrowane znaczniki czasowe odpowiadające momentom wdechu i wydechu umożliwiają korelację z sygnałem w celu analizy zależności między cyklem oddechowym a jego parametrami.

\begin{figure}[htbp]
    \centering
    \includegraphics[scale=0.17]{aplikacja.png}
    \caption{Interfejs aplikacji mobilnej}
    \label{fig:aplikacja_mobilna}
\end{figure}

Zarejestrowany zapis ulega filtracji dolnoprzepustowej. Detekcja lokalnych maksimów realizowana jest poprzez analizę trzech kolejnych próbek sygnału, środkowy punkt klasyfikowany jest jako szczyt, jeśli jego wartość przewyższa oba sąsiednie. Akceptacja następnego piku wymagana upływu co najmniej 600 ms od poprzedniego wykrycia, ograniczając błędne rozpoznania wynikające z zakłóceń ruchowych. Zidentyfikowane ekstremum jest rejestrowane wraz ze znacznikiem czasu i wykorzystywane do dynamicznego wyznaczania częstości uderzeń serca BPM, zgodnie z równaniem (1) :

\begin{equation}
\text{BPM} = \frac{60}{\Delta t}
\label{eq:bpm}
\end{equation}
gdzie $\Delta t$  - średni odstęp czasowy między kolejnymi pikami sygnału PPG

\paragraph{Proces rejestracji sygnału}
Dane wykorzystane do trenowania i walidacji modelu detekcji szczytów fali zostały pozyskane przy użyciu zaprojektowanej aplikacji. Obrazy przechwytywano w czasie rzeczywistym w formacie YUV 4:2:0 o rozdzielczości 640×480. Z każdej klatki wyodrębniano składową luminancji, na podstawie której obliczano średnią wartość jasności pikseli. Otrzymany odczyt stanowił pojedynczą próbkę surowego sygnału fotopletyzmograficznego.

W celu zgromadzenia odpowiedniego zbioru danych zrealizowano serię pomiarów z udziałem sześciu osób. Każda sesja trwała 10 minut i obejmowała rejestrację sygnału w zróżnicowanych warunkach, uwzględniających drobne ruchy palca wpływające na zmienność przepływu krwi.

Transmisję danych pomiędzy smartfonem a komputerem przeprowadzono z wykorzystaniem protokołu WebSocket. Informacje przesyłano w pakietach zawierających pojedyncze punkty pomiarowe wraz z odpowiadającymi im znacznikami czasu. Częstotliwość próbkowania była zgodna z liczbą klatek wideo, wynoszącą około 30 Hz. Odebrane dane buforowano na komputerze i zapisywano w formacie CSV.

\newpage
\subsection{Filtracja sygnału – filtr Butterwortha}
Filtr Butterwortha opracowano jako rozwiązanie analogowe o maksymalnie płaskiej charakterystyce amplitudowej w paśmie przepustowym, minimalizującej oscylacje i zniekształcenia sygnału. Charakteryzuje się monotonicznym tłumieniem w paśmie zaporowym oraz łagodniejszym zboczem przejściowym niż w strukturach o wyższej selektywności, w tym Czebyszewa i eliptycznych \cite{8}. W implementacji cyfrowej przyjmuje postać układu o nieskończonej odpowiedzi impulsowej IIR, gdzie bieżąca próbka wyjściowa zależy od wartości wejściowych, jak i poprzednich stanów wyjściowych. Podejście to umożliwia realizację filtrów wyższych rzędów przy ograniczonej liczbie współczynników, wykorzystywanych w aplikacjach czasu rzeczywistego \cite{9}.

Do przetwarzania sygnału EKG wykorzystano cyfrowy filtr piątego rzędu o charakterystyce pasmowo-przepustowej, obejmującej zakres częstotliwości od 0,5 Hz do 45 Hz, eliminujący szumy oraz zakłócenia. Dolna granica pasma redukuje powolne zmiany w zapisie wywołane ruchem ciała lub niestabliną pozycją elektrod  \cite{10}. Natomiast górna tłumi zakłócenia sieciowe, elektromagnetyczne oraz mięśniowe  \cite{11}. Filtracja danych została przeprowadzona w oparciu o bibliotekę NeuroKit2.

\begin{figure}[htbp]
    \centering
    \includegraphics[width=0.76\linewidth]{Filtr_EKG.png} 
    \caption{Porównanie surowego i przefiltrowanego sygnału EKG}
    \label{fig:filtr_ekg}
\end{figure}

\newpage
Dla poprawy jakości sygnału PPG wykorzystano filtr  czwartego rzędu, działający w zakresie 0,5–5 Hz. Dolna granica pasma ogranicza zakłócenia spowodowane ruchem ciała czy niestabilnym kontaktem czujnika ze skórą, natomiast górna tłumi szumy urządzenia pomiarowego i interferencje optyczne  \cite{12}. Filtracja została zrealizowana z wykorzystaniem  modułu SciPy, zapewniając obróbkę przebiegu bez przesunięcia fazowego.

\begin{figure}[htbp]
    \centering
    \includegraphics[width=0.76\linewidth]{Filtr_PPG.png} 
    \caption{Porównanie surowego i przefiltrowanego sygnału PPG}
    \label{fig:filtr_ppg}
\end{figure}


\subsection{Modele detekcji}
\subsubsection{Model do wykrywania załamków R}
Opracowano sieć neuronową łączącą warstwy splotowe i rekurencyjne, przeznaczoną do detekcji szczytów R w zapisie EKG. Stanowi ona pierwszy etap w wyznaczaniu interwałów RR i obliczaniu parametrów zmienności rytmu serca. Model przetwarza jednowymiarowe sygnały napięcia elektrycznego podzielone na fragmenty po 256 próbek, stanowiących podstawową jednostkę analizowanego przebiegu.

Część splotowa sieci obejmuje cztery warstwy konwolucyjne 1D, których parametry zestawiono w Tabeli~\ref{tab:conv_layers}. Warstwy te odpowiadają za ekstrakcję cech z sygnału wejściowego, rozszerzając jego reprezentację poprzez zwiększenie liczby kanałów od 16 do 128 za pomocą filtrów o rozmiarach 5 i 3. Dla każdego przekształcenia zastosowano normalizację BatchNorm1d oraz nieliniową funkcję aktywacji LeakyReLU. Redukcja wymiarowości jest realizowana za pomocą operacji MaxPooling1D z jądrem o rozmiarze 2, skracającej długość sekwencji z 256 do 16 próbek wzdłuż osi czasowej oraz zwiększającej odporność modelu na szum i zakłócenia. Wynikowa macierz o wymiarach [128 × 16] stanowi wejście dla jednokierunkowej LSTM. Otrzymany wektor jest spłaszczany i przekazywany do warstwy z 128 neuronami, a następnie do bloku końcowego zawierającego 256 elementów, reprezentujące kolejne próbki sygnału wejściowego.

\newpage
Dane wyjściowe z warstwy LSTM są przetwarzane przez moduły liniowe, odwzorowujące wyekstrahowane cechy w przestrzeń logitów. Każdy element wektora odpowiada prawdopodobieństwu obecności szczytu R w danej próbce, umożliwiając binarną klasyfikację w każdym punkcie czasowym przebiegu.


\begin{table}[h!]
\centering
\caption{Parametry warstw konwolucyjnych sieci}
\label{tab:conv_layers}
\begin{tabular}{|l|c|c|c|c|c|}
\hline
\textbf{Warstwa} & \textbf{Wejście} & \textbf{Wyjście} & \textbf{Filtr} & \textbf{Padding} & \textbf{Aktywacja} \\
\hline
Conv1D-1 & 1   & 16  & 5 & 2 & LReLU+BN \\
Conv1D-2 & 16  & 32  & 5 & 2 & LReLU+BN \\
Conv1D-3 & 32  & 64  & 3 & 1 & LReLU+BN \\
Conv1D-4 & 64  & 128 & 3 & 1 & LReLU+BN \\
\hline
\end{tabular}
\end{table}

Model wytrenowano w trybie nadzorowanym na podstawie sygnałów pozyskanych z czujnika Polar H10. Do oznaczenia rzeczywistych załamków R w przefiltrowanym zapisie EKG wykorzystano algorytm Pan–Tompkins. Każdej próbce przypisano etykietę binarną wskazującą obecność lub brak lokalnego maksimum, umożliwiając sieci klasyfikację wzorców odpowiadających rzeczywistym szczytom R. Proces uczenia przeprowadzono w partiach po 32 elementy z wykorzystaniem funkcji straty BCEWithLogitsLoss. Do optymalizacji zastosowano algorytm Adam przy stałej wartości learning rate wynoszącej 0,0001.

Efektywność zaprojektowanej sieci neuronowej oceniono na podstawie wyników uzyskanych na niezależnym zbiorze danych, wykorzystując wytrenowane parametry modelu. Rezultaty klasyfikacji przedstawiono w postaci macierzy konfuzji w Tabeli~\ref{tab:conf_matrix}.

\begin{table}[ht]
\centering
\caption{Macierz konfuzji dla detekcji załamków R}
\label{tab:conf_matrix}
\begin{tabular}{|c|c|c|}
\hline
\textbf{Rzeczywiste / Predykcja} & \textbf{Brak szczytu } & \textbf{Szczyt } \\
\hline
Brak szczytu  & 231170 & 53 \\
\hline
Szczyt  & 83 & 2678 \\
\hline
\end{tabular}
\end{table}

Model poprawnie zaklasyfikował 231170 próbek niezawierających piku R oraz 2678 z jego rzeczywistą obecnością. Liczba fałszywie pozytywnych predykcji wyniosła 53, natomiast fałszywie negatywnych -- 83. Wartość miary F1 równa 0,9753 świadczy o wysokiej równowadze pomiędzy precyzją a czułością modelu, co jest kluczowe w automatycznej analizie sygnałów elektrokardiograficznych. Podstawowe metryki oceny jakości modelu przedstawiono w Tabeli~\ref{tab:metrics}.

\begin{table}[ht]
\centering
\caption{Parametry detekcji szczytów R}
\label{tab:metrics}
\begin{tabular}{|c|c|p{4.6cm}|}
\hline
\textbf{Metryka} & \textbf{Wartość} & \textbf{Opis} \\
\hline
Skuteczność & 96,99\% & Odsetek poprawnych klasyfikacji obecności lub braku piku R. \\
\hline
Błędne detekcje & 0,00\% & Odsetek próbek fałszywie zaklasyfikowanych jako zawierające pik R. \\
\hline
Pominięte załamki & 3,01\% & Odsetek próbek z niewykrytym rzeczywistym pikiem R. \\
\hline
\end{tabular}
\end{table}

\newpage
\subsubsection{Model do wykrywania szczytów fali}
Analogicznie do podejścia zastosowanego w analizie sygnału EKG, opracowano model konwolucyjny identyfikujący lokalne szczyty w przebiegu fotopletyzmograficznym. Stanowi on podstawę do wyznaczania interwałów międzyuderzeniowych IBI, wykorzystywanych przy estymacji wskaźników rytmu serca. Sieć przetwarza dane w postaci wektora zawierającego 50 próbek w pojedynczym kanale, reprezentujących zmiany objętości krwi w naczyniach obwodowych.

Architektura składa się z trzech kolejnych warstw splotowych 1D, odpowiedzialnych za detekcję lokalnych wzorców w przebiegu. W pierwszych dwóch zastosowano filtry o szerokości 5 próbek oraz rosnącą liczbę kanałów od 1 do 32, a każde przekształcenie uzupełniono o normalizację BatchNorm1d i nieliniową funkcję aktywacji ReLU. Końcowy blok z filtrem o rozmiarze 1 generuje jednowymiarowy wektor, którego wartości wyznaczane przez funkcję sigmoidalną są interpretowane jako prawdopodobieństwa wystąpienia piku w poszczególnych punktach przebiegu. Szczegółowe parametry warstw splotowych uwzględniono w Tabeli~\ref{tab:ppg_layers}.


\begin{table}[ht]
\centering
\caption{Parametry warstw konwolucyjnych sieci}
\label{tab:ppg_layers}
\begin{tabular}{|l|c|c|c|c|c|}
\hline
\textbf{Warstwa} & \textbf{Wejście} & \textbf{Wyjście} & \textbf{Filtr} & \textbf{Padding} & \textbf{Aktywacja} \\
\hline
Conv1D-1 & 1 & 16 & 5 & 2 & ReLU+BN \\
Conv1D-2 & 16 & 32 & 5 & 2 & ReLU+BN \\
Conv1D-3 & 32 & 1 & 1 & 0 & Sigmoid \\
\hline
\end{tabular}
\end{table}

Proces uczenia nadzorowanego, wykorzystany w analizie zapisu EKG rejestrowanego za pomocą pulsometru, został w analogiczny sposób zastosowany do danych PPG pozyskanych z urządzenia mobilnego. Etykietowanie punktów oparto na identyfikacji szczytów lokalnych w przefiltrowanym przebiegu, wykorzystując procedurę wyszukiwania punktów, których wartość przewyższa sąsiednie próbki. Każdemu elementowi sygnału przypisano oznaczenie binarne wskazujące obecność lub brak piku, umożliwiając klasyfikację odpowiadających wzorców. Sieć trenowano w partiach po 32 próbki, stosując funkcję straty BCELoss oraz algorytm optymalizacji Adam ze stałym współczynnikiem uczenia równym 0,001.

Skuteczność zaprojektowanego modelu oceniono na podstawie wyników uzyskanych na niezależnym zbiorze danych, przy zastosowaniu parametrów wyznaczonych podczas treningu. Wyniki klasyfikacji przedstawiono w formie macierzy konfuzji w Tabeli~\ref{tab:conf_matrix_ppg}.

\begin{table}[ht]
\centering
\caption{Macierz konfuzji dla detekcji pików fali}
\label{tab:conf_matrix_ppg}
\begin{tabular}{|c|c|c|}
\hline
\textbf{Rzeczywiste / Predykcja} & \textbf{Brak szczytu } & \textbf{Szczyt} \\
\hline
Brak szczytu  & 9610 & 8 \\
\hline
Szczyt  & 12 & 370 \\
\hline
\end{tabular}
\end{table}

\newpage
Dla sygnałów fotopletyzmograficznych sieć poprawnie rozpoznała 9610 próbek bez obecności fali oraz 370 z jej wystąpieniem. Niepoprawne predykcje odnotowano w 8 przypadkach fałszywie dodatnich oraz 12  fałszywie ujemnych. Otrzymana wartość miary F1, wynosząca 0,9737, jest zbliżona do wyników uzyskanych dla modelu EKG, potwierdzając wiarygodność zastosowanego rozwiązania w analizie obu typów sygnałów.
Podstawowe metryki oceny jakości architektury przedstawiono w Tabeli~\ref{tab:metrics_ppg}.

\begin{table}[ht]
\centering
\caption{Parametry detekcji szczytów fali}
\label{tab:metrics_ppg}
\begin{tabular}{|c|c|p{4.6cm}|}
\hline
\textbf{Metryka} & \textbf{Wartość} & \textbf{Opis} \\
\hline
Skuteczność & 97,12\% & Odsetek poprawnych klasyfikacji obecności lub braku piku fali. \\
\hline
Błędne detekcje & 0,00\% & Odsetek próbek fałszywie zaklasyfikowanych jako zawierające pik fali. \\
\hline
Pominięte załamki & 2,88\% & Odsetek próbek z niewykrytym rzeczywistym pikiem fali. \\
\hline
\end{tabular}
\end{table}

\section{Synchronizacja czasowa i analiza porównawcza parametrów HRV}

\subsection{Synchronizacja czasowa sygnałów}
Podczas rejestracji sygnałów EKG i PPG wykorzystywane są różne schematy zapisu znaczników czasowych. W elektrokardiogramie punkty wystąpienia załamków R początkowo określane są względem chwili rozpoczęcia akwizycji i zapisywane w sekundach jako czas względny. Podczas przetwarzania danych wartości te przekształcane są do formatu absolutnego UNIX, umożliwiając porównanie z przebiegiem PPG. W fotopletyzmografii detekcja szczytów fali rejestrowana jest bezpośrednio w czasie systemowym.

Dla ujednolicenia układu czasowego przekształca się dane EKG z czasu względnego na znaczniki UNIX, zgodnie z równaniem (2):
\begin{equation}
t_{\mathrm{UNIX}} = t_{\mathrm{rel}} + t_{0},
\end{equation}
gdzie $t_{\mathrm{UNIX}}$ - czas w formacie UNIX, $t_{\mathrm{rel}}$ – czas względny, a $t_{0}$ – początek akwizycji.

Przebiegi zostały przedstawione w jednej osi czasu, umożliwiając ich automatyczną synchronizację. Różnice między odpowiadającymi pikami wyznaczono zgodnie z równaniem (3):
\begin{equation}
PTT = t_{\mathrm{PPG}} - t_{\mathrm{ECG}},
\end{equation}
gdzie $t_{\mathrm{PPG}}$, $t_{\mathrm{ECG}}$ - momenty detekcji szczytów w sygnale PPG i EKG. 

%Wskaźnik PTT wykorzystywany jest do oceny poprawności synchronizacji sygnałów.%

\newpage
\subsection{Wskaźniki HRV}
Ujednolicenie osi czasowych umożliwia równoległą analizę odstępów kolejnych uderzeń serca w obu przebiegach. Dla sygnału EKG określa się interwały RR, odpowiadające odległościom między załamkami R, natomiast w PPG interwały międzyuderzenowe IBI. Na podstawie tych wartości wyznaczono standardowe parametry HRV, przedstawione w formie RR. Analogiczne obliczenia przeprowadzono dla odstępów IBI.

\noindent\textit{1) Średnia długość interwału:} 
Średnia arytmetyczna odstępów RR, wyrażona wzorem (4):
\begin{equation}
    Mean = \frac{1}{N} \sum_{i=1}^{N} RR_i
\end{equation}
gdzie $RR_i$ -- $i$-ty odstęp RR, a $N$ – liczba analizowanych odstępów.

\noindent\textit{2) Odchylenie standardowe odstępów NN:} 
Wielkość całkowitej zmienności rytmu serca, obliczana na podstawie wszystkich interwałów RR, wyrażona wzorem (5):
\begin{equation}
    SDNN = \sqrt{\frac{1}{N-1} \sum_{i=1}^{N} (RR_i - Mean)^2}
\end{equation}
gdzie $Mean$ – średnia długość interwału, $RR_i$ – $i$-ty odstęp RR, a $N$ – liczba analizowanych odstępów. \textbf{}

\noindent\textit{3) Pierwiastek kwadratowy z uśrednionych kwadratów różnic kolejnych odstępów NN:} 
Miara stosowana w ocenie krótkoterminowych wahań rytmu serca, wyrażona wzorem (6):
\begin{equation}
    RMSSD = \sqrt{\frac{1}{N-1} \sum_{i=1}^{N-1} (RR_{i+1} - RR_i)^2}
\end{equation}
gdzie $RR_i$ -- $i$-ty odstęp RR, a $N$ – liczba analizowanych odstępów.


 
\newpage
\begin{thebibliography}{1}
\bibitem{1}
 O. Faust, U. R. Acharya, H. Adeli, and A. Adeli, "Deep learning for healthcare applications based on physiological signals: A review"
\bibitem{2}
 Rajpurkar, P., Hannun, A. Y., Haghpanahi, M., Bourn, C., \& Ng, A. Y. (2017), "Cardiologist-level arrhythmia detection with convolutional neural networks"
\bibitem{3}
 Yildirim, O. (2018), "A novel wavelet sequence based on deep bidirectional LSTM network model for ECG signal classification. Computers in Biology and Medicine"
\bibitem{4}
 Elgendi, M. (2012), "On the Analysis of Fingertip Photoplethysmogram Signals. Current Cardiology Reviews"
\bibitem{5}
Chen, W., et al., "Deep learning and remote photoplethysmography: A comprehensive review" 
\bibitem{6}
Zanelli, S., et al., "Transfer learning of CNN-based signal quality assessment for photoplethysmography"
\bibitem{7} 
Polar Electro, "Polar H10 heart rate sensor," 2025
\bibitem{8} 
W. M. Laghari, M. Baloch, M. Mengal i S. Shah, "Performance Analysis of Analog Butterworth Low Pass Filter as Compared to Chebyshev Type-I Filter, Chebyshev Type-II Filter and Elliptical Filter"
\bibitem{9} 
S. Chakraborty, K. K. Jha i A. Patra, "Design of IIR Digital Highpass Butterworth Filter using Analog to Digital Mapping Technique"
\bibitem{10} 
G. Lenis, N. Pilia, A. Loewe, W. H. W. Schulze, and O. Dössel, "Comparison of Baseline Wander Removal Techniques considering the Preservation of ST Changes in the Ischemic ECG: A Simulation Study"
\bibitem{11} 
R. J. Martis, U. R. Acharya, H. Adeli, "Current methods in electrocardiogram characterization"
\bibitem{12} 
Pimentel, M.A.F., et al. (2016), "Toward a robust estimation of heart rate from wrist-type PPG signals"

\end{thebibliography}
\end{document}


